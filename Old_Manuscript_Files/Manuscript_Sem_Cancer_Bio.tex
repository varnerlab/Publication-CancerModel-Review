\documentclass[12pt]{article}
% Load packages
\usepackage{url}  % Formatting web addresses  
\usepackage{ifthen}  % Conditional 
\usepackage{multicol}   %Columns
\usepackage[utf8]{inputenc} %unicode support
\usepackage{amsmath}
\usepackage{amssymb}
\usepackage{epsfig}
\usepackage{epstopdf}
\usepackage{graphicx}
\usepackage[margin=0.1pt,font=footnotesize,labelfont=bf]{caption}
\usepackage{setspace}
%\usepackage{longtable}
\usepackage{colortbl}
%\usepackage{palatino,lettrine}
%\usepackage{times}
%\usepackage[applemac]{inputenc} %applemac support if unicode package fails
%\usepackage[latin1]{inputenc} %UNIX support if unicode package fails
\usepackage[wide]{sidecap}
%\usepackage[authoryear,round,comma,sort&compress]{natbib}
\usepackage[square,sort,comma,numbers]{natbib}
%\usepackage[authoryear,round]{natbib}
\usepackage{supertabular}
%\usepackage{simplemargins}
\usepackage{comment}
\usepackage{lineno}
\usepackage{hyperref}

\urlstyle{rm}

%\textwidth = 6.50 in
%\textheight = 9.5 in
%\oddsidemargin =  0.0 in
%\evensidemargin = 0.0 in
%\topmargin = -0.50 in
%\headheight = 0.0 in
%\headsep = 0.25 in
%\parskip = 0.15in
%\linespread{1.75}
\doublespace

%\usepackage{geometry}
\usepackage{fullpage}

%\bibliographystyle{plain}
%\bibliographystyle{plos2009}

\makeatletter
\renewcommand\subsection{\@startsection
	{subsection}{2}{0mm}
	{-0.05in}
	{-0.5\baselineskip}
	{\normalfont\normalsize\bfseries}}
\renewcommand\subsubsection{\@startsection
	{subsubsection}{2}{0mm}
	{-0.05in}
	{-0.5\baselineskip}
	{\normalfont\normalsize\itshape}}
\renewcommand\section{\@startsection
	{subsection}{2}{0mm}
	{-0.2in}
	{0.05\baselineskip}
	{\normalfont\large\bfseries}}	
\renewcommand\paragraph{\@startsection
	{paragraph}{2}{0mm}
	{-0.05in}
	{-0.5\baselineskip}
	{\normalfont\normalsize\itshape}}
\makeatother

%Review style settings
%\newenvironment{bmcformat}{\begin{raggedright}\baselineskip20pt\sloppy\setboolean{publ}{false}}{\end{raggedright}\baselineskip20pt\sloppy}

%Publication style settings

% Single space'd bib -
\setlength\bibsep{0pt}

\renewcommand{\rmdefault}{phv}\renewcommand{\sfdefault}{phv}

% Change the number format in the ref list -
\renewcommand{\bibnumfmt}[1]{#1.}

% Change Figure to Fig.
\renewcommand{\figurename}{Fig.}



% Begin ...
\begin{document}
\begin{titlepage}
{\par\centering\textbf{\Large Physical and Logical Models of Signal Transduction Processes in Cancer }}
\vspace{0.05in}
{\par \centering \large{ Holly A. Jensen, Katharine V. Rogers, and Jeffrey D. Varner$^{*}$}}
\vspace{0.05in}
{\par \centering \large{School of Chemical and Biomolecular Engineering}}
{\par \centering \large{Cornell University, Ithaca NY 14853}}
\vspace{0.1in}
{\par \centering \textbf{To be submitted: Seminars in Cancer Biology}~\emph{}}
\vspace{0.5in}
{\par \centering $^{*}$Corresponding author:}
{\par \centering Jeffrey D. Varner,}
{\par \centering Assistant Professor, School of Chemical and Biomolecular Engineering,}
{\par \centering 244 Olin Hall, Cornell University, Ithaca NY, 14853} 
{\par \centering Email: jdv27@cornell.edu} 
{\par \centering Phone: (607) 255 - 4258}
{\par \centering Fax: (607) 255 - 9166}
\end{titlepage}
\date{}
\thispagestyle{empty}
\pagebreak
%%%%%%%%%%%%%%%%%%%%%%%%%%%%%%%%%%%%%%%%%%%%%%%%%%%%%%%%%%%%%%%%%%%%%%%%%%%%%%%%%%%%%%%%%%%%%%%%%%%%%%%%%%%
%%%%%%%%%%%%%%%%%%%%%%%%%%%%%%%%%%%%%%%%%%%%%%%%%%%%%%%%%%%%%%%%%%%%%%%%%%%%%%%%%%%%%%%%%%%%%%%%%%%%%%%%%%%
\section*{Abstract}


\pagebreak

\setcounter{page}{1}

\linenumbers

\section*{Introduction}

\subsection*{Cancer: Hundreds of diseases, few successes} 

Cancer, once considered a monolithic disease, is a vast repertoire of diseases divided into carcinomas (epithelial-originating cancers), sarcomas (connective tissue cancers), leukemias (blood cancers), lymphomas, myelomas, and mixed types like teratocarcinoma. All these diseases exhibit the \textquotedblleft Hallmarks of Cancer\textquotedblright \: as described by Hanahan and Weinburg in 2000 \cite{Hanahan2000}. These characteristics are (1) sustained proliferative signaling, (2) insensitivity to or evasion of growth-suppressive signals, (3) resistance to or evasion of apoptosis, (4) limitless renewal potential, (5) promotion of angiogenesis and (6) tissue invasion and metastasis. In a more recent review, to this list Hanahan and Weinburg added (7) altered metabolic signaling, and (8) resistance to immune destruction and resulting inflammation \cite{Hanahan2011}. These proposed hallmarks have in fact been criticized: it was pointed out that 5 of the original 6 (excluding ability to metastasize) are in fact characteristics of benign tumors as well \cite{Lazebnik2010}. Nonetheless, a general consensus exists that cancers do exhibit the above listed attributes, which can parsimoniously be described as notably harmful uncontrolled cell proliferation. Cancer can unfortunately arise in essentially any tissue type, resulting in \textquotedblleft hundreds of different cancers\textquotedblright \: \cite{CancerGovClasses}.  

The first and foremost cancer treatments were of course excision of the tumor. For breast cancer, the first treatment was radical mastectomy, pioneered by William Stewart Halsted in the late 19\textsuperscript{th} century, which left the patient horrifically disfigured \cite{Mukherjee2010}. Eventually knowledge of metastasis made obvious the limitations of surgery, and during the 20\textsuperscript{th}century surgeries involved removal of less tissue, and over time were combined with radiation therapy, and eventually chemotherapy. Radiation therapy had its own breakthrough in the early 20\textsuperscript{th} century, and was almost immediately realized to be capable of causing cancer as well. Currently, highly precise radiation therapies exist, such as . . . .. Some of the first chemotherapies were alkylating agents, discovered in the wake of studies related to mustard gas during the WWI era. The slough of chemotherapeutic agents isolated after that paved the way for the fervent, rigorous and determined application of every combination and cocktail of available chemotherapeutic compounds onto cancer patients in the 1970s and into the 1980s \cite{Mukherjee2010}. Also during this time, significant advances to our understanding and manipulation of DNA and RNA were achieved. The identification of the src gene in normal tissue (discovered in the wake of studies related to the Rous sarcoma virus which contained mutated src) initiated the search for mutant, dominant gain-of-function oncogenes and defunct tumor suppressors. The most notable culprits across different cancer types have been transmembrane growth factor receptors (EGFR, HER2), tyrosine kinases (Bcl-Abl), and nuclear hormone receptors (ER, RAR). This subsequently has resulted in an ongoing effort to develop drugs and therapies that target these specific components, in the hope to abate enhanced proliferative signaling, if not fully restore normal growth signaling. 

There are a multitude of small molecule inhibitors on the market, one of the first being imatinib (Gleevec), which inhibits the Bcl-Abl fusion protein and is used to treat GIST (gastrointerstinal stromal tumor) and CML (chronic myelogenous leukemic) \cite{Vanneman2012}. Beyond intracellular tyrosine kinase inhibitors, there are also small inhibitor compounds for growth factor receptors, such as gefitinib (Iressa) which targets EGFR and is given to patients with non-small cell lung cancer, and other signaling components, such as vemurafenib (Zelboraf) which targets a B-Raf mutation present in about half of all melanomas. Tamoxifen (binds estrogen receptor (ER), breast cancer), and tretinoin (retinoic acid, binds retinoic acid receptor (RAR), acute promyelocytic leukemia) are two examples of nuclear receptor targeted treatments. There are also approved drugs that are mTOR inhibitors such as temsirolimus (Torisel, approved for renal cell carcinoma) and everolimus (Afinitor, approved for advanced kidney cancer, advanced breast cancers, and others); angiogenesis inhibitors (VEGF receptor inhibitors) such as vandetanib (Calprelsa, medullary thyroid cancer) and sorafenib (Nexavar, for advanced renal cell and hepatocellular carcinoma); histone deacetylase (HDAC) inhibitors like vorinostat (Zolinza) and romidepsin (istodax) which treat cutanous T-cell lymphoma; proteosome inhibitors like bortezomib (Velcade) and carfilzomib (Kyprolis) for multiple myeloma; targeted antifolates like pralatrexate (Folotyn); and the list goes on. Monoclonal antibodies are another class of therapeutics that have rapidly gained momentum. The first therapeutic monoclonal antibodies were rituximab (Rituxan, non-Hodgkin lymphoma and CLL) and trastuzumab (Herceptin, HER2-positive breast cancer). Now there are many others, such as sunitinib (Sutent), which both inhibit tyrosine kinases and targets VEGF, and Cetuximab (Erbitux), which is approved for head and neck and colorectal cancer \cite{CancerTargetedTherapies}. Cancer vaccines are another up-and-coming therapy. Sipuleucel-T (Provenge) is an approved autologous vaccine for castration-resistant prostate cancer; the result is that T cell activity is simulated against the protein PAP (prostatic acid phosphatase) which is expressed by prostate cancer cells \cite{Vanneman2012}. Many therapies approved for one cancer type are being investigated as treatments for other cancers. 

However, targeted therapies are proving less promising than previously anticipated. Of all anticancer agents tested in the preclinical setting, only 5\% are successfully licensed after making it to Phase III clinical testing \cite{Hutchinson2011}. This low success rate is mainly due to poor candidate selection in the preclinical arena, which arises from shortcomings on how cancer therapies are pursued. Individual cell lines do not represent whole cancers; mouse xenografts do not reflect the human case; treatments are tested as monotherapeutics rather than combination therapies. Bevacizumab (Avastin) was a previously approved angiogenesis inhibitor for breast cancer treatment, until the FDA revoked approval in 2011 (despite slowing metastatic growth, it did not help patients live longer or improve prognosis, and had some harmful side effects) \cite{Pollack2011}. Emergent resistance is also a major obstacle in cancer therapy, and can arise in response to chemotherapeutic agents, kinase inhibitors, hormonal agents and immunomodulatory treatments. Some HER2-positive patients do not respond to trastuzumab \cite{Hutchinson2011}. In some cases chemotherapeutic combinations have been successful at overcoming resistance developed in response to single agents; but often cancerous cells exhibit cross-resistance to alternative compounds, or are \textit{de novo} resistant to treatment \cite{Garraway2012}. Resistance, especially in relation to kinase inhibitors, is often associated with a acquired mutation(s) in the intended target?examples include the emergence of a mutation in Bcl-Abl in chronic myelogenous leukemia (CML) cells treated with imatinib, mutation in PML-RAR in acute promyelocytic leukemia (APL) cells treated with retinoic acid, and an EGFR mutation in gefitinib-treated non-small cell lung cancer. These mutations are likely not produced by the treatment per se but exist in subpopluations that are then positively selected \cite{Garraway2012}. However, such acquired mutations are not the whole story of resistance. Genetic alterations can arise in signaling factors upstream or downstream of the target. Enhanced ERK activation results from MEK1 mutation or a mutant NRas that acts through c-Raf; both of these mechanisms render B-Raf inhibition ineffectual \cite{Garraway2012}. Bypass mechanisms result when a downstream effector of the target is activated via an alternative pathway, or when feedback inhibition is inadvertently relieved \cite{Garraway2012}. Beyond this, sometimes no resulting mutations can be identified. Even pathway-independent resistance is possible, such as altered tumor angiogenesis in response to both EGFR inhibitors and therapeutic anti-EGFR antibody \cite{Garraway2012}. Resistance is, overall, poorly understood at present.

\subsection*{The Systems Biology Effort} 

In the end, it is evident that cancers are diseases epitomized by dysregulation of entire networks. Although all cancers have a genetic basis, with genome alterations either inherited or induced from external factors (viruses, carcinogens, radiation), holistic understanding at the genetic, intracellular, tissue and extracellular (tumor environment) and physiological level is still necessary to develop successful future therapeutics for such a complex disease. A systems biology approach is required. In 2004 the NCI launched the Integrative Cancer Systems Biology Program, which included funding for the 12 Cancer Systems Biology Centers at top research institutions that comprise the core of the program. The effort recognizes the fact that a systems biology approach using both experimental and computational methods is required to make significant headway on our understanding of cancers. Below we address some of the progress toward using both experimental and computational methods for cancer systems biology.

\section*{Experimental Tools for Cancer Systems Biology}

\subsection*{Cancer cell lines: Successful surrogates?}

The first established, continuous human cell lines of any kind were understandably cancer cell lines, starting with the infamous HeLa cell line isolated from the cervical tissue of Henrietta Lacks in 1951 \cite{Masters2002,Lucey2009}. This cell line proliferates so robustly that it later caused a controversial outbreak when found to have cross-contaminated other cell line cultures in many laboratories, and worst still, many cell lines were in fact misidentified HeLa cells \cite{Masters2002,Lucey2009}. Following the development of defined culture media, cell culture became a mainstay of cancer research. The first breast cancer cell line (BT-20) was isolated in 1958 \cite{Holliday2011,LASFARGUES1958}. There were over 200 human normal and malignant hematopoietic cell lines being studied by the mid-1970s \cite{Nilsson1975,Koeffler1980}. Today, established cancer cell lines available for study number well into the thousands, many of which are available from cell bank repositories such as ATCC (American Type Cell Culture), DSMZ (German Collection of Microorganisms and Cell Cultures) and ECACC (European Collection of Cell Cultures).

In the late 1980s, the National Cancer Institute (NCI) initiated the development of a cancer cell line panel, now called NCI60 \cite{Shoemaker2006}. The project actually began as the Lung Cancer Drug Discovery Project to pool various lung cancer cell lines for drug screening. However, due to the need for comparative controls, the project evolved to include cell lines that represented breast, colon, central nervous system, leukemia, lung, melanoma, ovarian, and prostate cancers. The need to systematically perform cytotoxic screens for all 60 cell lines subsequently prompted the development of new assays that could be executed on a large scale \cite{Shoemaker2006}. The NCI60 panel (in whole or in part) was used to screen not only synthetic compounds but the amassed collection of natural products collected by the Development Therapeutics Program DTP \cite{Shoemaker2006}. Moving into the 1990s, screening identified compounds that underwent \textit{in vivo} testing. By 2000 the NCI60 panel had evolved from NCI drug-discovery screen into a research tool for the entire cancer research community. Another panel that was also developed around the same time was JFCR39: 39 cancer cell lines (some of which were subsets of NCI60) pooled together by the Japanese Foundation for Cancer research.

Some of the key questions and concerns that were addressed during the development of the NCI60 panel still remain. \textquotedblleft How accurate are cancer cell lines?\textquotedblright \: was the title of an article published in Nature a few years ago \cite{Borrell2010}. Do cancer cell lines retain significant semblance to their derived tissue type, and do these cell lines actually provide decent surrogates for the \textit{in vivo} disease? A main concern is that tumor-derived lines, once removed from a highly complex environment (structurally and biochemically different, surrounded by other cell types), undergo genetic changes during adaption as a monoculture. In short, the established culture may differ from the original tissue. A recent study \cite{Domcke2013} compared cell line data (from the Cancer Cell Line Encyclopedia database) to primary tumor sample data (from the Cancer Genome Atlas database) and revealed that while some widely-used ovarian cancer cell lines are not accurate models, other less-used ovarian cell lines bear much better semblance to ovarian tumors. It has also been pointed out \cite{Sharma2010} that studies using tissue-specific cell line panels reveal that a repertoire of cancer cell lines are indeed representative of their respective heterogeneous cancer tissues \cite{Neve2006,Sos2009,Lin2008}. 

The \textquotedblleft next generation\textquotedblright \: cancer cell line panel is CMT1000, short for the Center of Molecular Therapeutics 1000, and consists of 1,200 cancer cell lines \cite{Sharma2010}. This panel overcomes the shortcomings of NCI60 which, while extremely valuable for generic cytotoxic compound screening, is less effective for targeted therapy screens due to the lack of representation of cancers from other tissues, and does not account for the degree of heterogeneity that in facts exists even within cancers from the same tissue type. The CMT1000 panel is further asserting that cancer cell lines do indeed recapitulate clinical findings \cite{Sharma2010}. CMT1000 screenings are also shifting the traditional outlook, from one of importance of tissue similarity, to that of genetic similarity in terms of response to kinase inhibitors \cite{Sharma2010}. In the end, it is necessary to employ both primary cells and established cell lines in cancer research. Primary cells provide what is and will be ultimately needed in an era of personalized medicine where there are hundreds of versions of cancer: not a model per se, but the exact cancer cells themselves. But primary cells are accompanied by issues with heterogeneity of collected tissue samples, which makes (for example) sequencing their genomes and identifying mutations expensive since thousands of comparisons to normal tissue must be made \cite{Borrell2010}. Cancer cell lines, and notably cancer cell line panels, remain a mainstay for foundational research, drug screening and identifying genotypes.
 
\subsection*{Beyond cell lines: 3D culture and animal models} 

Cancer cell lines will remain at the forefront of cancer research. However, 2D monolayer cultures grown on plastic simply cannot account for the structural and architectural aspects of solid tumors, and typically do not account for microenvironment effects such as gradients in oxygen/nutrients or the influence of other cell types. Meanwhile animal models (xenografts and/or genetically engineered models) tend to be more expensive and time-consuming, not to mention may fail to be accurate representations for the human case \cite{Worp2010}. Thus 3D cancer culture models allow for a realistic, structurally-relevant human cancer cell system. Cancer cells grown in 3D cultures compared to 2D cultures can exhibit (1) changes in rates of proliferation, (2) changes in cell morphology and (3) changes in gene expression. 3D cell culture scaffolds also provide a controllable microenvironment in which different cell types can be introduced to create a heterogenous tissue that mimics the \textit{in vivo} environment. 3D culture systems can take many forms, such as multilayer systems, matrix-embedded cell cultures and multicellular tumor spheroids \cite{Sharma2010}. Supported 3D culture systems can be hydrogels, polymer scaffolds, or better yet biological components (collagen, extracellular matrix components).
	
We are still in the early days of 3D cell culture, facing issues with reproducibility and sensitivity, in addition to the fact that 3D cultures are not yet competitive with current high-throughput methods (limitations of automation and cost). Nonetheless, 3D cell culture is an incredibly useful emerging technology \cite{Kunz-Schughart2004}, and several commercially available systems already exist. BD Biosciences has several products (such as Matrigel) related to 3D cell culture. The Perfecta3D 384-Well Hanging Drop Plates are a readily available culture system for spheroid-based 3D cell cultures. The GravityPLUS technology developed by InSphero is similar to the hanging drop approach, but eliminates the need for scaffold material. Happy Cell\textregistered\: ASM (Biocroi) is a product developed by Dr. Anthony Davies (Trinity College, Dublin), that is compatible with high-throughput liquid handlers because the suspension polymer is a liquid, rather than a solid gel, scaffold or micropatterned device \cite{Glaser2013}.
	
One of the most compelling advances that has implications for regenerative medicine, cancer research and personalized medicine (personalized drug screening) is the growth of ex vivo organoids. These micro-organs, pioneered by Dr. Hans Clevers, recapitulate the structure and architecture of their derived tissue type. Cultured organoids of the gut epitheliam (?miniguts?), containing all major intestinal cell types, have already been successful \cite{Sato2009,Schwank2013}. Drs. Hans Clevers and Johannes Bos of UMCU (University Medical Center Utrecht) are now the leaders of a multinational team effort, involving American labs and European institutes, and funded by Stand Up to Cancer (SU2C), to develop organoids for colon, prostate and pancreatic tumors \cite{Krol2013}.
	
The most common preclinical animal research model across academia and industry is undoubtedly the mouse. Mouse cancer models either involve subcutaneous or orthotopic xenografts of human cells, or now, genetically engineered mouse models (GEMMs). The ability of mouse models to recapitulate human cancers and reflect meaningful human responses to cancer therapeutics has been questioned \cite{Worp2010,Hutchinson2011}. Nonetheless, mouse xenografts and GEMMs remain valuable tools for investigating the pharmacological properties of therapeutics \cite{Sharma2010}. Patient-derived tumor tissue can grow when implanted into immunodeficient mice: in the subcutaneous space of athymic nude mice, SCID (severe combined immune deficient) mice and NOD/SCID (nonobese diabetic SCID) mice \cite{Jin2010}. The first genetically modified cancer mouse model was OncoMouse \cite{Mukherjee2010}. The NCI?s Mouse Models of Human Cancer Consortium is continuing the effort of employing mouse models to better understand cancer; the NCI CaMOD (Cancer Model Database at http://cancermodels.nci.nih.gov) contains information on mice, rat and other animal cancer models that have been developed \cite{Hanahan2007}.

\subsection*{High-throughput technologies for systems biology} 

Our understanding of cancer and other diseases has progressed hand-in-hand with the technology available to study them, and no technology stands out more for high-throughput cancer studies than next generation sequencing (NGS). Whether congenital, spontaneous or induced by mutagens, all mutations are genetic, therefore the first level of understanding cancer is understanding the genetic aberrations. While microarrays were a golden standard in their day, DNA and RNA analysis using mciroarrays was only possible for known sequences, had high noise and little dynamic range. Next-generation sequencing (NGS), comprehensively reviewed in \cite{Soon2013}, has enabled much more than highly paralyzed processing of sequences. The decreasing cost and emerging technology of NGS entails powerful methods beyond those of the straight forward DNAseq, RNAseq, and ChIPseq. Epigenetic characteristics, such as histone modification and methylation patterns, actively transcribed or translated sequences, nascent transcripts, distinct structural chromosome territories?all can be monitored and mapped with increasing accuracy using combinatorial cross-linking, chemical treatment, immunoprecipitation and other methods combined with NGS.
	
Current second generation sequencing platforms include SOLiD, which uses sequential ligation of oligonuleotide probes, the Illumina HiSeq2000 which uses reversible dye terminators, and Roche/454 which uses pyrosequencing. Typically second generation sequencing platforms require several orders of magnitude of DNA or RNA, which when collected from many pooled cells, can potentially result in \textquotedblleft averaged\textquotedblright \: and thus misleading data. Looking forward, in stark contrast to second generation platforms which rely on massively parallel systems, emergent third generation sequencing technology aims to sequence a single DNA molecule without any previous amplification, i.e. true single molecular sequencing (tSMS) \cite{Pareek2011}. Current well-established single cell methods include flow cytometry, laser capture microdissection, but these have not reached a promising scale. Single-cell sequencing would address the fact that normal and malignant tissues are often quite heterogeneous cell systems, and single-cell sequencing has been successfully used to identify increased genetic complexity () as well as tumor subpopulations () \cite{Soon2013}. Third generation platforms have the potential to rapidly increase sequence rates by eliminating DNA template amplification (and thus eliminating error from replication as well), exploiting the naturally high processivity of DNA polymerase, and eradicating the use of detection nucleotides altogether \cite{Pareek2011}. Third generation tSMS platforms include the HeliScope Single Molecule Sequencer (Helicos), the PacBio\textregistered\: RS II (Pacific Biosciences), the nucleotide-free Nanopore DNA sequencers GridION and MinION (Oxford Nanopore Technologies\textregistered), the unreleased VisiGen Biotechnologies real-time sequencer, and the Ion Personal Genome Machine (PGM\textsuperscript{TM}) sequencer using PostLight technology (by Ion Torrent) \cite{Pareek2011}. 
	
Cancer genome sequencing has been instrumental in defining genetic commonalities for various cancers, point mutations (SNVs), indels, inversions amplifications, translocations, chromothripses, etc. The availability of NGS has resulted in a multitude of consortia efforts to collect and pool this sequencing information., and consequently a numerous comprehensive databases are now readily available. For example, the dbGaP (Genome and Phenotype Database) catalogues variant SNPs with a range of conditions. The Cancer Cell Line Encyclopedia (CCLE) project, launched in 2012 by the Broad Institute and Novartis \cite{Barretina2012}, aims to characterize the genetic profiles of around 1,000 cancer cell lines. The CCLE can be accessed here: http://www.broadinstitute.org/ccle/home. Meanwhile the Cancer Genome Atlas (TGCA, http://cancergenome.nih.gov/), begun by the NCI and the National Human Genome Research Institute (NHGRI) in 2005 and expanded in 2009, aims to sequence the genomes of more than 20 different cancer types. The COSMIC (Catalogue of Somatic Mutations in Cancer) database at http://cancer.sanger.ac.uk/cancergenome/projects/cosmic/), one of the projects at the Wellcome Trust Sanger Institute, aims to identify and index somatic variations and mutations in various cancer types. OncoMap
	
The GridION platform mentioned previously can be adapted to protein analysis as well. As stated before, cancer is a disease that operates at multiple levels that must be understood holistically. Moving away from genome, epigenome and transcriptome levels, and toward intracellular signaling, high-throughput technologies are required at the proteome and signalosome level as well. Launched in 2003, the Human Protein Atlas project (http://www.proteinatlas.org) maps protein expression profiles of various normal and malignant human tissues, and ultimately aims to enable systematic development of specific antibodies and identify potential biomarker candidates \cite{Ponten2011}. The project employs the Dako Autostainer Plus Staining System for staining cell microarray sections, followed by automated digital slide-scanning using the Aperio Technology?s Scanscope XT \cite{Fagerberg2011}. Today?s microplate readers now include much more advanced systems, such as the FLIPR Tetra High Throughput Cellular Screening System, which enables rapid fluorescent or luminescent cell assay analysis in microwell plate format. High Content Screening (HCS) refers to the automation of large scale cell biology techniques and most often refers to automated confocal microscopy, an improvement to traditional small-scale microscopy techniques. However HCS can refer to other novel high throughput instrumentation as well as the integrated data analysis software required \cite{Abraham2004}. The Thermo Scientific\textsuperscript{TM} ArrayScan\textsuperscript{TM} XTI High Content Analysis (HCA) Reader is a combination high content fluorescent and transmitted light imager, plate reader and flow cytomer (http://www.thermoscientific.com/en/products/cellular-imaging-analysis.html?ca=highcontent). One review \cite{Zanella2010} tabulates other HCS instrumentation and informatics software now available. Flow cytometry, already a high throughput technology, has also seen improvements which include new devices, such as HyperCyt\textregistered\: (an autosampler that connects to flow cytometers to improve sampling capacity) \cite{Edwards2004,Black2011}, and adapting new protocols to current devices, such as using FACS (fluorescence-activated cell sorting) to screen enzyme activity \cite{Yang2009}. Meanwhile high-throughput mass spectrometry (MS) for proteome mapping presents significant challenges due to the existence of isoforms, splice variants, single nucleotide polymorphisms (SNPs) and hundreds of potential post-translational modifications \cite{Nilsson2010}. However, improved search algorithms and protein databases, such as Swiss-Prot/UNiProt, combined with an effort to standardize the way proteomic data is stored (the HUGO Proteomics Standard Initiative), are making MS methods more comparable to high throughput proteomics \cite{Nilsson2010}.
	
It is worthwhile to step away from benchtop instrumentation and mention a few smaller devices that are enabling a systematic approach to cancer research and cancer drug screening. Smaller \textquotedblleft high throughput\textquotedblright \: screening technologies are most often microfluidic devices that permit highly parallel data collection, while simultaneously increasing accuracy by using single-cell resolution. There are microfluidic devices for cell culture \cite{Wu2010}, microscopy \cite{Di-Caprio2013} and even cytometry \cite{Ehrlich2011}; there are micro droplets systems for selecting genes \cite{Fallah-Araghi2012} or monitoring enzyme activity \cite{Agresti2010,Zagnoni2011}. Another up-and-coming device is the tissue microarray (TMA), which is a paraffin block containing potentially hundreds of embedded tissue samples \cite{Jawhar2009}. This method provides a bridge between monocultured cell lines and animal models, and is proving favorable due to reproducibility, reduced time and cost, and limiting the amount of sample needed to be taken from the original tissue. The process of creating the tissue array has already been automated \cite{Jawhar2009}. What now remains, as proposed by Dr. Mike Shuler (Cornell University) is to create a \textquotedblleft mouse on a chip\textquotedblright \: or \textquotedblleft human on a chip\textquotedblright, in which representative tissue samples are combined with a microfluidic circulatory system. Such efforts are already underway at several institutions \cite{Baker2011}.

\section*{Computational Models}

\subsection*{Current approaches: Kinetic Models} 

Cancer involves the dysregulation of multiple signaling pathways in which computational modeling can be applied to understand complex network responses. One of the most common modeling approaches for signal transduction networks is through a set of coupled ordinary differential equations, using mass action kinetics \cite{Aldridge2006}. The equations used are derived from established chemical and physical theory \cite{Aldridge2006}. ODE kinetic models often require extensive prior knowledge of network structure, rate constants and initial conditions \cite{Kholodenko2012}. Even with this, the ability of ODE models to capture dynamics makes it a particularly useful tool in studying cell signaling. Lauffenburger and coworkers developed early biophysical and kinetic models of epidermal growth factor receptor signaling in fibroblastic cells and Interleukin 2 receptor signaling in T-cells \cite{Starbuck1992, Forsten1994}. Both models provided key insights into critical network parameters involved in cell proliferation. A more potent ligand for the epidermal growth factor receptor was later developed using these models \cite{Reddy1996}. Later, ordinary differential equation (ODE) models were developed focusing on the downstream signaling due to the presence of growth factors and its effect on cell fate decisions \cite{Kholodenko1999, Schoeberl2002}. 

Almost two decades later, multiple cancer signaling systems have been studied using an ODE framework. DNA damage response was studied with a p53/MdM2 network model containing negative and positive feedback loops leading to p53 oscillations \cite{Ciliberto2005}. Apoptosis through caspase regulatory networks has been explored using experimental training data from HeLa cells \cite{Rehm2006,Albeck2008a,Albeck2008}. Analysis of the mammalian NF-$\kappa\beta$ system by Hoffman and coworkers, predicted bimodal signal characteristics of the IkappaB-NF-kappaB signaling module \cite{Hoffmann2002}. Other important cancer signaling systems that have been explored include RTK and MAPK (mitogen-activated protein kinase) cascades \cite{Bhalla2002, Schoeberl2002,Borisov2009,Chen2009}, JAK-STAT signaling \cite{Swameye2003,Vera2008}, and Wnt signaling \cite{Leeuwen2007}, \cite{Leeuwen2009}, \cite{Kim2007}. 

As cancer often involves dysregulatoin of multiple signaling pathways involving crosstalk and feedback, larger systems need to be developed to provide a more accurate portrayal of cancer systems. For example, a model by Kim \textit{et al.} discovered a positive feedback loop between the Wnt and ERK pathways \cite{Kim2007}. A model by Borisov \textit{et al.} predicted increased mitogenic signaling due to crosstalk between insulin and EGF signaling networks \cite{Borisov2009}. This model predicted and experiments confirmed that inhibition of PIP3 positive feedback abolished the increased mitogenic signaling due to insulin. Tasseff \textit{et al.} developed a model to reveal new targets for androgen independent prostate cancer \cite{Tasseff2010}. Initially, treatments for prostate cancer target the androgen receptor signaling pathway, but often the cancer progresses into an androgen-independent phenotype. The model includes androgen receptor signaling as well as crosstalk between the androgen receptor and the MAPK pathway, itself a predicted mechanism for the development of androgen-independent prostate cancer \cite{Feldman2001}. As more complete knowledge between signal transduction networks and interactions between signaling pathways become known, computational models will become even more important in aiding in the understanding of these complex networks.

Often, the option of adding all known biology to computational models of cancer signal transduction networks is not possible. Model size is often limited due to the difficulty in solving for unknown model parameters. Gadkar \textit{et al.} showed that it was often impossible to identify all the parameters in signal transduction methods even with near perfect knowledge of the system and high frequency sampling \cite{Gadkar2005}. This idea indicates that experimental design may need to play a larger role in generating better training and validation data sets for model identification \cite{Apgar2010}. This suggests that ODE models may also require significant mechanistic prior knowledge of the underlying biology. Although, more then a decade ago Bailey suggested that qualitative and quantitative knowledge of complex biological systems could be achieved in the absence of complete structural and parameter knowledge \cite{Bailey2001}. Later, Sethna and coworkers showed that the sensitivity of model behavior and predictive ability was dependent on only a few parameter combinations, a characteristic common to multiparameter signaling models referred to as sloppiness \cite{Daniels2008}. Thus, even with limited parameter information reasonable model predictions could be possible. Taking advantage of this sloppy model hypothesis, we have developed techniques for parameter identification using ensembles of deterministic models. A multi-objective optimization approach, Pareto optimal ensemble techniques (POETs), explores parameter space while accounting for uncertainty and conflicts in experimental training data \cite{Song2010}. We have proposed that the sloppiness of biological models may be a source of cell-to-cell \cite{Lequieu2011} or even patient-to-patient heterogeneity \cite{Luan2010}. Recently, cell-to-cell heterogeneity has been explored through Bayesian techniques of parameter estimation \cite{Hasenauer2011} %\cite{Kalita2011}. 
A dynamic ensemble of networks can portray a population of cells since the operational biochemical pathways are often context-specific \cite{Creixell2012}.    

\subsection*{Current approaches: Logical Models}

Other computational methods have been utilized for cancer networks, due to the size constraint in kinetic models. One such method, is a logic based model. Logic based models are graphical representations of signaling networks in which the nodes of the graph represent proteins and the edges represent interactions \cite{Morris2010}. The components are connected with logical gates, where each gate relates inputs to outputs. A subset of these, known as boolean logic models, divide network components into one of two activation states (on and off). These models are simpler then mechanistic models, but one limitation is relating nonbinary data to two distinct activation states (on and off) \cite{Kholodenko2012}. Multiple approaches have been added to logic-based models to allow for the modeling of intermediate states of activity. For example, in multistate discrete models additional levels between 0 and 1 are specified \cite{Morris2010}. Additionally, fuzzy logic has been utilized to allow for component values to range continuously from 0 to 1 \cite{Morris2010}. 

Logic based models are important to cancer research, because they are typically simpler to solve then mechanistic models and less a priori knowledge is required. In the earliest known logical-based biological model, Kauffman used discrete logic to model gene regulation \cite{Kauffman1969}. In 2000, Huang and Ingber were one of the first to develop a logic-based model of a cell-signaling network. The model explored different fates (proliferation, differentiation, apoptosis) of individual cells due to external stimuli and specific molecular cues \cite{Huang2000}. Due to the large scale of cancer networks, many logical models of biological networks have been developed. A Boolean model, containing 94 nodes and 123 interactions, of T cell receptor signaling predicted unexpected signaling events that were experimentally validated \cite{Saez-Rodriguez2007}. Using a Boolean logic model of EGFR signaling, qualitative model predications were compared to high throughput data from human hepatocytes and liver cancer cells (HepG2) \cite{Samaga2009}. The use of logical models may also be able to give some insight into medical applications. Boolean models of the early response of liver cells to cytokines and small molecule inhibitors were developed by training against primary human hepatocytes and 4 liver cancer cell lines \cite{Saez-Rodriguez2009,Saez-Rodriguez2011}. These Boolean models, in combination with high-throughput data, predicted distinct models for each cell type with models clustering into normal and diseased sets. Heiser and coworkers utilized a Pathway Logic model to determine EGFR-MAPK signaling in 30 breast cancer lines \cite{Heiser2009}. The model identified Pak1 as a key node in regulating the MAPK cascade when over-expressed. Through experimental validation they determined that Pak1 over-expressing luminal breast cancer cell lines have increased sensitivity to MEK inhibition.  

\subsection*{Multiscale Models for Cancer Systems Biology}

Cancer is a multi-scale disease. As mentioned previously, holistic understanding at the genetic, intracellular, tissue and extracellular (tumor environment) and physiological level is required in order to develop successful future therapeutics for such a complex disease. The next step from \textit{in silico} intracellular signaling network models is multi-scale models that dynamically recapitulate tumor cell migration (metastasis), angiogenesis, and other microenvironment effects like cell-cell interactions and/or nutrient delivery. Multiscale mathematical angiogenesis models (reviewed in \cite{Qutub2009}) were developed as early as the 1970s. Now, there is a vast array of literature for modeling multiscale systems using different methods. Multiscale models \cite{Deisboeck2011} are generally either continuous (employing partial differential equations), discrete (employing stochastic methods), or hybrid models. Continuum models \cite{Swanson2003,Massey2012} are advantageous for describing an entire range of spatial and temporal properties, but often result in a population-averaged view of the modeled tumor. Discrete methods \cite{Chavali2008,Hattne2005} are better suited for revealing the emergent properties of (for example) individual cell decisions, but these methods tend to be less scalable \cite{Chakrabarti2012}. The most prevalent hybrid method is agent based modeling (ABM) in which discrete autonomous \textquotedblleft agents\textquotedblright \: (which exist is different states) act within a spatially and temporally continuous environment. A set of rules determines how the continuous environment influences the agents, and/or vice versa.
	
Multi-scale ABM models \cite{Kaul2013} can follow a top-down or bottom-up approach and may (bottom-up) or may not (top-down) be coupled to an intracellular signaling dynamics. Top-down approaches employ coarse-grained empirical rules to describe global system characteristics, and are easy to implement with software packages like NetLogo \cite{Sklar2007} or CompuCell \cite{Andasari2012}. Meanwhile bottom-up approaches are becoming more and more popular for modeling biological complexity, using signaling networks to guide the action of agents \cite{Chakrabarti2012}. Avascular cancer growth was modeled by Ferreira \textit{et al.} (2002) using nutrient reaction-diffusion, cell proliferation and death, and cell motility; the model qualitatively captured commonly observed morphologies for primary tumors \cite{Ferreira2002}. In 2005 Jiang \textit{et al.} described another model for avascular mulitcellular tumors, which employed a Boolean network at the subcellular level, a lattice Monte Carlo model for proliferation and adhesion at the cellular level, and reaction-diffusion dynamics for extracellular chemicals concentrations \cite{Jiang2005}. CancerSim is an agent based simulation developed by Abbott \textit{et al.} (2006) that recapitulates the \textquotedblleft Hallmarks of Cancer\textquotedblright \: put forth by Hanahan and Weinberg \cite{Abbott2006}. Implemented in CancerSim are cells that can develop very crude and simplified \textquotedblleft mutations\textquotedblright \: (characteristics), such as \textquotedblleft evade apoptosis\textquotedblright \: or \textquotedblleft ignore growth inhibit\textquotedblright. The simulation typically results in a heterogeneous cell population and predicts that when mutation rates are low, certain pathways will dominate \cite{Abbott2006}. In 2009 \cite{Wang2009} Wang \textit{et al.} expanded upon their earlier work \cite{Wang2009} to develop a 3D model of non-small-cell lung cancer that also incorporated previously omitted TGF$\beta$, and showed that targeted monotherapy could be ineffective. Perfahl \textit{et al.} (2011) reported a bottom-up 3D lattice-based model of vascular tumor growth that incorporated subcellular signaling mechanisms and stochastic elements like endothelial tip cell emergence \cite{Perfahl2011}. 


\subsection*{Using Models to Develop Drug Targets}

The use of computational models of cancer networks to discover new drug targets, particularly in resistant cancers, and to allow for personalized treatment are relatively new ideas. Often, primary targets for cancer types are known, but crosstalk and feedback of other signaling pathways can lead to resistance even in the presence of inhibitors (reference). In particular, one receptor system which has been extensively modeled is that of the epidermal growth factor receptor (EGFR) (reviewed elsewhere) \cite{Wiley2003}. EGFR is a receptor that is overexpressed in many human tumors including breast, lung, head and neck, colorectal, and more \cite{Salomon1995}. Other ErbB family members have also been studied (reference). Some progress has recently been made in using computational models to discover novel targets in cancers where ErbB signaling is important. For example, Schoeberl \textit{et al.} developed MM-121, a human monoclonal antibody against ErbB3, after revealing through sensitivity analysis that ErbB3 was a key node in their computational model of the ErbB signaling network \cite{Schoeberl2009}. Currently many ErbB receptor inhibitors are used as treatments in several cancers, although resistance is an issue (reference). Models have been developed to find new targets in resistant cancers, including cancers resistant to trastuzumab \cite{Faratian2009,Sahin2009}. Faratian \textit{et al.} developed a kinetic model which included AKT/MAPK cross-talk, PTEN, HER2/HER3 dimerization and inhibition, and receptor tyrosine kinase (RTK) inhibitor binding \cite{Faratian2009}. The model hypothesized that PTEN expression levels predict cell sensitivity to RTK inhibitors and was experimentally confirmed using primary breast cancer samples. Sahin \textit{et al.} developed a Boolean logic model to find novel targets for trastuzumab resistant breast cancer \cite{Sahin2009}. The model, which combined ErbB signaling with G1/S transition of the cell cycle, identified c-MYC as a potential new target. 

Computational models can also be utilized to determine combination treatments for cancer and possibly even preferred treatment regimens. Recently, Kirouac \textit{et al.} developed a multiscale systems model of HER2 positive breast cancer to predict combination therapies \cite{Kirouac2013}. Signal transduction events were modeled using a quantitative logic framework, while tumor growth kinetics and feedback regulation were modeled using an ODE framework. Model predictions in combination with experiments in mice, showed dual inhibition of HER3 and HER2 as a treatment for HER2 positive breast cancer. Additionally, a signal transduction model of EGFR in colon cancer cells predicted dual inhibition of MEK and EGFR as a treatment \cite{Klinger2013}. Decreased tumor growth due to this dual inhibition was experimentally confirmed in a xenograft tumor model of KRAS-mutant colon cancer. A mass action kinetic model of insulin-like growth factor (IGF-1) signaling in breast cancer cells predicted optimal drug combinations \cite{Iadevaia2010}. Computational modeling may also be useful in determining drug regimens. A recent study by Lee \textit{et al.} predicted that pretreatment with an EGFR inhibitor sensitizes a subset of triple-negative breast cancer cells to DNA-damaging chemotherapy \cite{Lee2012}. 


\section*{Conclusions}


\clearpage

\bibliography{References_Review}
\bibliographystyle{nature}


%\end{comment}


\end{document}

